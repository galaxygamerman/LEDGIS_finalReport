\chapter{Introduction}
In our modern world where most of our data is digital including government documents and records, making sure that these records are completely reliable, secure and are conveniently easy to reach is very important.  Around the world many governments are experimenting with blockchain technology implementation including the countries like Estonia and Georgia to name a few. Traditional methods of record keeping usually have problems with being prone to data breaches or tampering of data. These issues can pose major problems for a government that is tasked with handling these huge amounts of data.\\
\par
\noindent In the digital age, where cyber threats and information warfare are growing concerns, the legal infrastructure must evolve to meet modern standards of resilience and accountability. Legal systems that continue to rely on outdated or paper-based record-keeping are increasingly vulnerable to inefficiencies and breaches. The move toward digitization is not just a technological upgrade—it is a foundational shift that influences how justice is delivered, how citizens access their rights, and how governments maintain legitimacy in the public eye.\\
\par
\noindent However, blockchain technology provides an excellent solution to these problems. By leveraging Hyperledger, which is an open-source set of tools built to support business transactions as well as governments and utilizing blockchain’s immutable ledger and its decentralized architecture, governments can substantially improve the transparency, security, and efficiency in legal record-keeping. This system also works well in protecting legal documents, since fraud in such cases is either highly unlikely or not easy to perform through the use of smart contracts.\\
\par
\noindent Beyond just securing records, blockchain-based systems offer operational advantages. For instance, the ability to trace every change or access event on a record creates a fully auditable trail, which can be crucial in legal proceedings. Such transparency is not only useful in fighting corruption or malpractice, but also in establishing clear accountability among public officials and institutions. Moreover, automated workflows using smart contracts reduce the need for intermediaries, which in turn cuts down administrative costs and red tape.\\
\par
\noindent As digital governance continues to mature, blockchain technology could become a backbone of secure civic infrastructure. The vision is not just limited to legal records—it could extend to voting systems, citizen identity management, and inter-agency data exchange. In this broader context, the current project serves as a foundational step toward a transparent and digitally empowered legal ecosystem that sets the stage for future innovation.\\
\par\noindent Our project seeks to explore the possibility and feasibility of successfully incorporating this system into legal frameworks to develop government legal records that cannot be tampered with, improving law enforcement procedures and promoting greater security in the judicial process. The research looks into other forms of implementation similar to this and the difficulties that are there to implement such a system. Overall, based on the general research a blockchain model seems most appropriate considering its potential impact and widespread use in the judicial system.